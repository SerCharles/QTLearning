\documentclass{article}
\usepackage{CJKutf8}
\title{第一次实验报告}
\author{软71 沈冠霖 2017013569}
\begin{document}
\begin{CJK}{UTF8}{gbsn}
\maketitle
\section{实验目标}本次实验有四个目标:首先,自己实现字符串,栈,链表,哈希表四个基本类,并且尽量做成模板类来复用。其次,能够读取几十万长度的词库,并且进行保存,散列变换。之后,能够对符合要求的网页进行解析,提取其标题,时间,来源,正文。最后,对正文进行中文分词。这一切都是为了构建之后的新闻检索与推荐系统。
\section{实验环境} Win10,VS2017下进行开发。

\section{抽象数据定义}
\paragraph{字符串}定义了一个字符串类NCharString,可以进行空字符串构造(空构造函数),通过char来构造(重载构造函数),赋值(m_SetValue),复制(m_Duplicate),截取(m_CutString),拼接(f_Concat,m_PushBackChar,m_PushBackString),查找子串(m_FindSubString),比较(m_Compare)等功能。
\paragraph{链表}定义了查找(查找第i个位置的m_SearchAtPlace,查找第一个和给定数据相等的m_SearchEqual),在给定位置后插入(m_InsertAtPlace),删除给定位置(m_DeleteAtPlace),构造和析构功能。还继承了一个字符串链表NCharStringLink,有一个m_PushBack函数。
\paragraph{栈}定义了一个栈NStack,基于链表,里面存储标签和标签内的内容,便于进行解析。可以实现入栈(mm_PushBack),出栈(m_PopBack),判断是否为空(m_JudgeEmpty).
\paragraph{哈希表}定义了一个29万长度的哈希表,可以把一个我自定义类型的字符串链表转换成哈希,也可以读入一个我自定义的字符串来搜索其位置。此处有引用代码[1]。
\paragraph{具体模块}定义了一个常数类Constant和一个全局常数类变量g\_Constant,负责读入开始的几个数字,并存储4个数字,24结果等信息。定义了一个类NumberBase来存储每种情况的结果,及其使用的数字,还有其父母情况。定义了一个类NumberList来生成,存储所有情况,并生成算式。
\section{网页解析}

\end{CJK}
\end{document}
