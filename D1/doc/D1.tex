\documentclass{article}
\usepackage{CJKutf8}
\title{D1文档}
\author{软71 沈冠霖 2017013569}
\begin{document}
\begin{CJK}{UTF8}{gkai}
\maketitle
\section{T1第一问} 
\subsection{代码} 
\paragraph{}定义了一个类Member,将string类的姓名和int类的年龄设置为const,private,来保证其封装性和不可修改。
\paragraph{}重载了两个构造函数:一个是无参数构造函数,设置名称为?,年龄为0;另一个支持传入string和int型的姓名,年龄参数。都用初始化表初始化姓名年龄,来保证其满足const要求。
\paragraph{}用Member的友元函数重载了cout,来保证能用cout输出。
\subsection{运行方式与结果}用linux命令行运行以下指令: g++ -o test.exe main.cpp Member.cpp Member.h,这样就生成可执行文件。再执行即可。输出结果如下。\\
\\
\\
The name of the player is Zhang San, and his age is 22\\
The name of the player is Li Si, and his age is 19\\
The name of the player is Wang Wu, and his age is 18\\
The name of the player is Zhao Liu, and his age is 24\\
The name of the player is ?, and his age is 0\\
\\
\section{T1第二问} 
\subsection{代码} 
\paragraph{} 比起第一问,第二问多了一个Member的友元类MemberList,能用里面的vector存储member元素。其构造函数有两种重载形式:默认的空vector形式,可以读入数组和长度来初始化vector形式。析构函数则实现了清空vector,防止内存泄漏。
\paragraph{} 而且MemberList类用一个成员函数重载了数组中括号,实现了搜索年龄的功能。具体功能是如果有叫这个名字的队员,则输出搜索到的第一个队员的年龄。否则输出-1.
\subsection{运行方式与结果}运行方式同上。输出结果如下。\\
\\
\\
The name of the player is Zhang San, and his age is 22\\
The name of the player is Li Si, and his age is 19\\
The name of the player is Wang Wu, and his age is 18\\
The name of the player is Zhao Liu, and his age is 24\\
The name of the player is ?, and his age is 0\\
22\\
19\\
18\\
24\\
-1\\
\\
\section{T2}定义了一个基类Shape,设置了一个虚函数getarea,还有protected变量area,这些是各种具体图形通用的属性和方法。同时派生了三个子类:Circle,Rectangle,Square。分别覆盖了getarea函数。 运行结果和题目完全一致。
\section{T3}在hpp文件Max.hpp中声明与定义了模板函数Max(因为模板函数定义声明必须在一起),用一个模板函数实现了题目要求。
\end{CJK}
\end{document}
