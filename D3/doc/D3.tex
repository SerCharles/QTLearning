\documentclass{article}
\usepackage{CJKutf8}
\title{D3文档}
\author{软71 沈冠霖 2017013569}
\begin{document}
\begin{CJK}{UTF8}{gkai}
\maketitle
\section{T3} 
\subsection{代码} 
\paragraph{}定义了TestTimeArray类来随机生成前两个数组,用clock\_t变量记录时间。用6个子类实现了六种循环顺序,覆盖了数组运算函数,利用动态多态性来比较六种方法的时间。
\subsection{结果}
\paragraph{数据范围$50^{3}$}结果如下:\\
The 1 th Group, which uses the sequense of ijk, spends 934 ms per calculation.\\
The 2 th Group, which uses the sequense of ikj, spends 985 ms per calculation.\\
The 3 th Group, which uses the sequense of jik, spends 1141 ms per calculation.\\
The 4 th Group, which uses the sequense of jki, spends 1049 ms per calculation.\\
The 5 th Group, which uses the sequense of kij, spends 1265 ms per calculation.\\
The 6 th Group, which uses the sequense of kji, spends 1343 ms per calculation.\\
\paragraph{数据范围$100^{3}$}结果如下:\\
The 1 th Group, which uses the sequense of ijk, spends 5954 ms per calculation.\\
The 2 th Group, which uses the sequense of ikj, spends 7787 ms per calculation.\\
The 3 th Group, which uses the sequense of jik, spends 6132 ms per calculation.\\
The 4 th Group, which uses the sequense of jki, spends 7107 ms per calculation.\\
The 5 th Group, which uses the sequense of kij, spends 12975 ms per calculation.\\
The 6 th Group, which uses the sequense of kji, spends 11912 ms per calculation.\\
\paragraph{数据范围$200^{3}$}结果如下:\\
The 1 th Group, which uses the sequense of ijk, spends 40287 ms per calculation.\\
The 2 th Group, which uses the sequense of ikj, spends 64228 ms per calculation.\\
The 3 th Group, which uses the sequense of jik, spends 40798 ms per calculation.\\
The 4 th Group, which uses the sequense of jki, spends 62277 ms per calculation.\\
The 5 th Group, which uses the sequense of kij, spends 207003 ms per calculation.\\
The 6 th Group, which uses the sequense of kji, spends 247932 ms per calculation.\\
\paragraph{数据范围$400^{3}$}结果如下:\\
The 1 th Group, which uses the sequense of ijk, spends 251845 ms per calculation.\\
The 2 th Group, which uses the sequense of ikj, spends 534599 ms per calculation.\\
The 3 th Group, which uses the sequense of jik, spends 280314 ms per calculation.\\
The 4 th Group, which uses the sequense of jki, spends 534119 ms per calculation.\\
The 5 th Group, which uses the sequense of kij, spends 1882023 ms per calculation.\\
The 6 th Group, which uses the sequense of kji, spends 2108728 ms per calculation.\\
\subsection{结论}
\subsubsection{规律}循环为ijk顺序的是最快的,循环为kij,kji的很慢。而且数据范围越大,不同顺序之间速度差距越大。
\subsubsection{分析}
\paragraph{}首先,三种循环的时间效率都是$n^{3}$,完全一样。其次,三种循环进行的比较次数也是一样的,因为都是从0到n循环,比较$n+n^{2}+n^{3}$次。那么造成这种不同的主要差距应该是调取数据的速度。
\paragraph{}三维数组在内存空间以链状连续存储。元素[i][j][k]和[i][j][k+1]距离为1;[i][j][k]和[i][j+1][k]距离为n;[i][j][k]与[i+1][j][k]距离为$n^{2}$;这样,在整个计算过程中,指针在内存中移动距离越小,计算就应该越快。
在ijk顺序中,绝大多数的时间指针一次移动距离都是1。而在kji,kij中,指针绝大多数时间移动距离都是$n^{2}$。这样读取数据时间前者就比后者小。当n越大的时候,指针移动距离差距更大,运行效率差距就更大了。因此我推测有上面的结果的原因主要是读取总内存时间不同。
\end{CJK}
\end{document}
